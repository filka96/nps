\documentclass[14pt,a4paper]{scrartcl}
\usepackage[left=25mm, right=15mm, top=20mm, bottom=20mm, bindingoffset=0cm]{geometry}
\usepackage[utf8]{inputenc}
\usepackage[english,russian]{babel}
\usepackage[T1]{fontenc}
\usepackage{times}
\usepackage{listings}
\usepackage[usenames,dvipsnames]{color}

\lstset{
language=R,
basicstyle=\tiny\ttfamily,
numbers=left,
numberstyle=\tiny\color{Blue},
stepnumber=1,
numbersep=5pt,
backgroundcolor=\color{white},
showspaces=false,
showstringspaces=false,
showtabs=false,
frame=single,
rulecolor=\color{black},
tabsize=2,
captionpos=b,
breaklines=true,
breakatwhitespace=false,
keywordstyle=\color{RoyalBlue},
commentstyle=\color{YellowGreen},
stringstyle=\color{ForestGreen}
}

\begin{document}
    % Титульник
    \begin{titlepage}
        \begin{center}
            \large
            МИНИСТЕРСТВО ОБРАЗОВАНИЯ И НАУКИ\\ РОССИЙСКОЙ ФЕДЕРАЦИИ\\
            \textbf{Федеральное агентство по образованию}
            \vspace{0.5cm}
            ТВЕРСКОЙ ГОСУДАРСТВЕННЫЙ УНИВЕРСИТЕТ\\
            \vspace{0.25cm}
            Математический факультет\\
            Кафедра математического анализа
            \vfill
            \textsc{ВЫПУСКНАЯ РАБОТА БАКАЛАВРА}\\[5mm]
            {\LARGE Разработка модуля непараметрического анализа данных в пакете R\\[2mm]}
            \bigskip
        \end{center}
        \vfill
        \hfill
        \begin{minipage}{0.4\textwidth}
            Автор:\\
            Семенков Филипп Андреевич\\ \\
            Научный руководитель:\\
            Сидорова Оксана Игоревна\\
        \end{minipage}
        \vfill
        \begin{center}
            Тверь, 2018 г.
        \end{center}
    \end{titlepage}

    % Оглавление
    \newpage
    \tableofcontents

    % Содержание
    \newpage
    %\part[<short title>]{<title>}
    %\chapter[<short title>]{<title>}
    %\section[<short title>]{<title>}
    %\subsection[<short title>]{<title>}
    %\subsubsection[<short title>]{<title>}
    %\paragraph[<short title>]{<title>}
    %\subparagraph[<short title>]{<title>}

    \section{Введение}
    \subsection{Актуальность}
    \subsection{Задача}
    \subsection{Подзадачи}

    \section{Практика}
    \subsection{Пример решения задачи}
    \subsection{Выделение необходимых функций}
    \subsection{Наполнение приложухи необходимыми функциями}
    \subsection{Деплой приложухи}

    \section{Я - большой раздел}
    \subsection{Я - подраздел поменьше}
    \subsubsection{А я - самый маленький раздельчик}
    \paragraph{Ну совсем крошечный кусочек}

    % Перечисление
    \section{Перечисление}
    \begin{enumerate}
        \item item 1
        \item item 2
        \item item 3
    \end{enumerate}

    % Исходный код R
    \section{Исходный код R}
    \begin{lstlisting}[language=R]
        library(foreign)

        foo <- rnorm(100)
        # writing a function
        bar <- apply(foo, 1, function(x){
            y <- sqrt(x)
            cat(paste('The result is ', x )))
        })
        bar

        str(bar)

        foo + bar
    \end{lstlisting}

    % Список литературы
    \newpage
    \begin{thebibliography}{9}
        \bibitem{Knuth-2003}Кнут Д.Э. Всё про. \newblock --- Москва: Изд. Вильямс, 2003. 550~с.
    \end{thebibliography}

\end{document}
