\documentclass[14pt,a4paper]{scrartcl}
\usepackage[left=25mm, right=15mm, top=20mm, bottom=20mm, bindingoffset=0cm]{geometry}
\usepackage[utf8]{inputenc}
\usepackage[english,russian]{babel}
\usepackage[T1]{fontenc}
\usepackage{times}
\usepackage{listings}
\usepackage[usenames,dvipsnames]{color}

\lstset{
language=R,
basicstyle=\ttfamily,
numbers=left,
numberstyle=\color{Blue},
stepnumber=1,
numbersep=5pt,
backgroundcolor=\color{white},
showspaces=false,
showstringspaces=false,
showtabs=false,
frame=single,
rulecolor=\color{black},
tabsize=4,
captionpos=b,
breaklines=true,
breakatwhitespace=false,
keywordstyle=\color{RoyalBlue},
commentstyle=\color{YellowGreen},
stringstyle=\color{ForestGreen}
}

\begin{document}
    % Титульник
    \begin{titlepage}
        \begin{center}
            \large
            МИНИСТЕРСТВО ОБРАЗОВАНИЯ И НАУКИ\\
            РОССИЙСКОЙ ФЕДЕРАЦИИ\\
            \textbf{Федеральное агентство по образованию}
            \vspace{0.5cm}
            ТВЕРСКОЙ ГОСУДАРСТВЕННЫЙ УНИВЕРСИТЕТ\\
            \vspace{0.25cm}
            Математический факультет\\
            Кафедра математического анализа
            \vfill
            \textsc{ВЫПУСКНАЯ РАБОТА БАКАЛАВРА}\\[5mm]
            {\LARGE Разработка модуля непараметрического анализа данных в пакете R\\[2mm]}
            \bigskip
        \end{center}
        \vfill
        \hfill
        \begin{minipage}{0.4\textwidth}
            Автор:\\
            Семенков Филипп Андреевич\\
            \newline
            Научный руководитель:\\
            Сидорова Оксана Игоревна\\
        \end{minipage}
        \vfill
        \begin{center}
            Тверь, 2018 г.
        \end{center}
    \end{titlepage}

    % Оглавление
    \newpage
    \tableofcontents

    % Содержание
    % Теория
    \newpage
    \section[Теоретическая часть]{Теоретическая часть}
    \subsection[Постановка задачи]{Постановка задачи}
    \subsection[Актуальность]{Практическая ценность}
    \subsection[Требования к программе]{Требования к программе}
    \subsection[Новизна]{Задел на будущее}
    \subsection[Пользовательский интерфейс]{Пользовательский интерфейс}
    \subsection[Программный интерфейс приложения]{Программный интерфейс приложения (API)}
    \subsection[Развёртывание приложения]{Развёртывание приложения}

    % Практика
    \newpage
    \section[Практическая часть]{Практическая часть}
    \subsection[Выбор технологий]{Выбор технологий}
    \subsection[Пользовательский интерфейс]{Пользовательский интерфейс}
    \subsection[Программный интерфейс приложения]{Программный интерфейс приложения (API)}
    \subsection[Развёртывание приложения]{Развёртывание приложения}

    \newpage
    \section[Заключение]{Заключение}

    % Список литературы
    \newpage
    \addcontentsline{toc}{section}{Список литературы}
    \begin{thebibliography}{9}
        \bibitem{Knuth-2003}Кнут Д.Э. Всё про. \newblock --- Москва: Изд. Вильямс, 2003. 550~с.
    \end{thebibliography}
\end{document}
