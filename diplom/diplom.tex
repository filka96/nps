\documentclass[14pt,a4paper]{scrartcl}
\usepackage[left=25mm, right=15mm, top=20mm, bottom=30mm, bindingoffset=0cm]{geometry}
\usepackage[utf8]{inputenc}
\usepackage[english,russian]{babel}
\usepackage[T1]{fontenc}
\usepackage{times}
\usepackage{listings}
\usepackage[usenames,dvipsnames]{color}

\lstdefinelanguage{Dockerfile}{morekeywords={FROM, RUN, CMD, LABEL, MAINTAINER, EXPOSE, ENV, ADD, COPY, ENTRYPOINT, VOLUME, USER, WORKDIR, ARG, ONBUILD, STOPSIGNAL, HEALTHCHECK, SHELL}, morecomment=[l]{\#}, morestring=[b]"}

\lstset{language=R, basicstyle=\tiny\ttfamily, numbers=left, numberstyle=\tiny\color{Blue}, stepnumber=1, numbersep=5pt, backgroundcolor=\color{white}, showspaces=false, showstringspaces=false, showtabs=false, frame=single, rulecolor=\color{black}, tabsize=4, captionpos=b, breaklines=true, breakatwhitespace=false, keywordstyle=\color{RoyalBlue}, commentstyle=\color{YellowGreen}, stringstyle=\color{ForestGreen}}

\begin{document}
    % Титульник
    \begin{titlepage}
        \begin{center}
            \large
            МИНИСТЕРСТВО ОБРАЗОВАНИЯ И НАУКИ\\
            РОССИЙСКОЙ ФЕДЕРАЦИИ\\
            \textbf{Федеральное агентство по образованию\\}
            \vspace{0.5cm}
            ТВЕРСКОЙ ГОСУДАРСТВЕННЫЙ УНИВЕРСИТЕТ\\
            \vspace{0.25cm}
            Факультет прикладной математики и информатики\\
            Кафедра математической статистики и системного анализа\\
            \vfill
            \textsc{ВЫПУСКНАЯ РАБОТА БАКАЛАВРА}\\[5mm]
            {\LARGE Разработка модуля непараметрического анализа данных в пакете R\\[2mm]}
            \bigskip
        \end{center}
        \vfill
        \hfill
        \begin{minipage}{0.4\textwidth}
            Автор:\\
            Семенков Филипп Андреевич\\
            \newline
            Научный руководитель:\\
            Сидорова Оксана Игоревна\\
        \end{minipage}
        \vfill
        \begin{center}
            Тверь, 2018 г.
        \end{center}
    \end{titlepage}

    % Оглавление
    \newpage
    \tableofcontents

    % Содержание
    % Теория
    \newpage
    \section[Теоретическая часть]{Теоретическая часть}
    \subsection[Введение]{Введение}
    В учебном плане факультета прикладной математики и кибернетики на кафедре математической статистики и системного
    анализа мы изучаем довольно большое количество различных дисциплин.

    В курсе непараметрической статистики мы изучаем
    нормальные и экспоненциальные распределения, гамма - распределения, логарифмически нормальные и другие.
    Используются такие методы, которые позволяют изучить данные из выборки небольшого объема с переменными,
    о распределении которых мало или ничего не известно. Непараметрические методы не основаны на оценке
    параметры (например, среднее или стандартное отклонение) при описании распределения выборок
    значения. Поэтому эти методы иногда также называются параметрами, свободными или свободно распространяемыми.

    \subsection[Постановка задачи]{Постановка задачи}
    Для решения задач непараметрической статистики используются различные статистичиские пакеты например пакет R.
    В пакете R существует большое количество библиотек в которых содержатся необходимые для решения подобных задач функции.

    \subsection[Актуальность]{Практическая ценность}
    \subsection[Требования к программе]{Требования к программе}
    \subsection[Новизна]{Задел на будущее}
    \subsection[Пользовательский интерфейс]{Пользовательский интерфейс}
    \subsection[Программный интерфейс приложения]{Программный интерфейс приложения (API)}
    \subsection[Развёртывание приложения]{Развёртывание приложения}

    % Практика
    \newpage
    \section[Практическая часть]{Практическая часть}
    \subsection[Выбор технологий]{Выбор технологий}
    \subsection[Пользовательский интерфейс]{Пользовательский интерфейс}
    \subsection[Программный интерфейс приложения]{Программный интерфейс приложения (API)}
    \subsection[Развёртывание приложения]{Развёртывание приложения}
    \subsection[Конечная структура приложения]{Конечная структура приложения}

    % Примеры
    \newpage
    \section[Примеры решения задач]{Примеры решения задач}
    \subsection[Какя-то задача]{Какя-то задача}

    % Заключение
    \newpage
    \section[Заключение]{Заключение}

    % Список литературы
    \newpage
    \addcontentsline{toc}{section}{Список литературы}
    \begin{thebibliography}{9}
        \bibitem{Knuth-2003}Кнут Д.Э. Всё про. \newblock --- Москва: Изд. Вильямс, 2003. 550~с.
    \end{thebibliography}
\end{document}

% Перечисление
%\section{Перечисление}
%\begin{enumerate}
%\item item 1
%\item item 2
%\item item 3
%\end{enumerate}

% Исходный код R
%\newpage
%\section{Исходный код R}
%\begin{lstlisting}[language=R]
%library(foreign)
%foo <- rnorm(100)
%# writing a function
%bar <- apply(
%foo, 1, function(x){
%y <- sqrt(x)
%cat(
%paste('The result is ', x )
%)
%}
%)
%bar
%str(bar)
%foo + bar
%\end{lstlisting}
