\documentclass[14pt,a4paper]{scrartcl}
\usepackage[left=25mm, right=15mm, top=20mm, bottom=30mm, bindingoffset=0cm]{geometry}
\usepackage[utf8]{inputenc}
\usepackage[english,russian]{babel}
\usepackage[T1]{fontenc}
\usepackage{times}
\usepackage{listings}
\usepackage[usenames,dvipsnames]{color}
\usepackage{enumitem}
\usepackage{graphicx}
\graphicspath{ {images/} }

\lstdefinelanguage{Dockerfile}{morekeywords={FROM, RUN, CMD, LABEL, MAINTAINER, EXPOSE, ENV, ADD, COPY, ENTRYPOINT, VOLUME, USER, WORKDIR, ARG, ONBUILD, STOPSIGNAL, HEALTHCHECK, SHELL}, morecomment=[l]{\#}, morestring=[b]"}

\lstset{language=R, basicstyle=\tiny\ttfamily, numbers=left, numberstyle=\tiny\color{Blue}, stepnumber=1, numbersep=5pt, backgroundcolor=\color{white}, showspaces=false, showstringspaces=false, showtabs=false, frame=single, rulecolor=\color{black}, tabsize=4, captionpos=b, breaklines=true, breakatwhitespace=false, keywordstyle=\color{RoyalBlue}, commentstyle=\color{YellowGreen}, stringstyle=\color{ForestGreen}}

\begin{document}
    % Титульник
    \begin{titlepage}
        \begin{center}
            \large
            МИНИСТЕРСТВО ОБРАЗОВАНИЯ И НАУКИ\\
            РОССИЙСКОЙ ФЕДЕРАЦИИ\\
            \textbf{Федеральное агентство по образованию\\}
            \vspace{0.5cm}
            ТВЕРСКОЙ ГОСУДАРСТВЕННЫЙ УНИВЕРСИТЕТ\\
            \vspace{0.25cm}
            Факультет прикладной математики и информатики\\
            Кафедра математической статистики и системного анализа\\
            \vfill
            \textsc{ВЫПУСКНАЯ РАБОТА БАКАЛАВРА}\\[5mm]
            {\LARGE Разработка модуля непараметрического анализа данных в пакете R\\[2mm]}
            \bigskip
        \end{center}
        \vfill
        \hfill
        \begin{minipage}{0.4\textwidth}
            Автор:\\
            Семенков Филипп Андреевич\\
            \newline
            Научный руководитель:\\
            Сидорова Оксана Игоревна\\
        \end{minipage}
        \vfill
        \begin{center}
            Тверь, 2018 г.
        \end{center}
    \end{titlepage}

    % Оглавление
    \newpage
    \tableofcontents

    % Содержание
    % Теория
    \newpage
    \section[Теоретическая часть]{Теоретическая часть}
    \subsection[Введение]{Введение}
    В учебном плане факультета прикладной математики и кибернетики на кафедре математической статистики и системного анализа мы изучаем довольно большое количество различных дисциплин.
    В курсе непараметрической статистики мы изучаем нормальное и экспоненциальное распределения, гамма - распределения, логарифмически нормальные и другие.
    Для решения задач непараметрической статистики используются методы, позволяющие изучать выборку небольшого объема с данными в которых содержатся такие параметры, о распределении которых мало известно или абсолютно ничего не известно.
    Такие методы не основаны на оценке параметров.

    \subsection[Практическая ценность]{Практическая ценность}
    Для решения таких задач используются различные статистические пакеты, такие как пакет R.
    В пакете R существует большое количество библиотек, содержащих готовые функции, которые помогут решить задачу.
    Для вызова подобных функций, необходимо проделать несколько шагов:

    \begin{itemize}[noitemsep]
        \item Установка пакета R
        \item Поиск библиотеки с необходимой функцией
        \item Установка библиотеки
        \item Написание программы для вызова функции
    \end{itemize}

    Установить сам интерпретатор R довольно просто.
    К тому же для этого языка существуют IDE.
    Самый популярный из них это RStudio.
    RStudio в общем упрощает написание программы на языке R.
    Он включает в себя редактор кода, инструменты для отладки и визуализации.
    Найти и установить необходимую библиотеку не так просто.
    Необходимые функции разбросаны по разным пакетам.
    А некоторые пакеты имеют специфические зависимости, на установку и настройку которых может потребоваться много времени.
    Да и для написания программы необходимы знания языка и время на изучение пакетов.
    Всё это натолкнуло меня на мысль о создании универсального модуля в котором будут собраны необходимые пакеты.

    \subsection[Задел на будущее]{Задел на будущее}
    Просто собрать необходимые пакеты в одну программу это сложная задача, но ещё сложнее предоставить к нужным функциям единый удобный интерфес, который можно будет использовать на различных устройствах.
    Для того, чтобы в любой момент можно было добавить или удалить функцию, поменять её описание и при этом не сломать всё остальное стоит разработать программный интерфейс приложения (API).
    API может помочь в наполнении программы функциями, а так же позволит использовать модуль не только в решении задач непараметрической статистики.
    В таком случае этот модуль будет полезен не только в рамках поставленной задачи.

    \subsection[Постановка задачи]{Постановка задачи}
    Цель: Разработать модуль непараметрического анализа данных в пакете R.\\
    Задачи:
    \begin{enumerate}
        \item Разработать требования к программе
        \item Выбрать и изучить стек технологий, согласно требованиям
        \item Разработать структуру приложения
        \item Написать приложение
    \end{enumerate}

    \subsection[Требования к программе]{Требования к программе}
    Требования к поведению программы:
    \begin{itemize}
        \item Модуль должен являтся приложением, написанным на языке R
        \item Модуль должен позволять решать задачи непараметрической статистики
        \item Модуль должен иметь пользовательский интерфейс
    \end{itemize}
    Требования к характеру поведения системы:
    \begin{itemize}
        \item Модуль должен иметь возможность расширения функционала
        \item Доступ к модулю может осуществлятся с помощью различных устройств
        \item Устройства могут иметь различные операционные системы
        \item Пользовательский интерфейс должен корректно отображаться на устройствах с различными экранами
    \end{itemize}

    \subsection[Пользовательский интерфейс]{Пользовательский интерфейс}
    \subsection[Программный интерфейс приложения]{Программный интерфейс приложения (API)}
    К тому же, по данным агентства 'We Are Social', возросло количество интернет-пользователей, большинство из них являются владельцами смартфонов\cite{Internet-statistic-2018}
    \subsection[Развёртывание приложения]{Развёртывание приложения}

    % Практика
    \newpage
    \section[Практическая часть]{Практическая часть}
    \subsection[Выбор технологий]{Выбор технологий}
    \subsection[Пользовательский интерфейс]{Пользовательский интерфейс}
    \subsection[Программный интерфейс приложения]{Программный интерфейс приложения (API)}
    \subsection[Развёртывание приложения]{Развёртывание приложения}
    \subsection[Конечная структура приложения]{Конечная структура приложения}

    % Примеры
    \newpage
    \section[Примеры решения задач]{Примеры решения задач}
    \subsection[Какя-то задача]{Какя-то задача}

    % Заключение
    \newpage
    \section[Заключение]{Заключение}

    % Список литературы
    \newpage
    \addcontentsline{toc}{section}{Список литературы}
    \begin{thebibliography}{9}
        \bibitem{Internet-statistic-2018}Social media use jumps in Q1 despite privacy fears
        \newblock --- https://wearesocial.com/uk/blog/2018/04/social-media-use-jumps-in-q1-despite-privacy-fears
    \end{thebibliography}
\end{document}

% Исходный код R
%\newpage
%\section{Исходный код R}
%\begin{lstlisting}[language=R]
%library(foreign)
%foo <- rnorm(100)
%# writing a function
%bar <- apply(
%foo, 1, function(x){
%y <- sqrt(x)
%cat(
%paste('The result is ', x )
%)
%}
%)
%bar
%str(bar)
%foo + bar
%\end{lstlisting}

%https://ru.sharelatex.com/learn/Inserting_Images